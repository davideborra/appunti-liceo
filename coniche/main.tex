\documentclass{article}
\usepackage[utf8]{inputenc}
\usepackage[italian]{babel} %for language elements
\usepackage{graphicx}   %for adding figures
\usepackage[a4paper, portrait, margin=2cm]{geometry}   %paper shape
\usepackage{array}      %table management
\usepackage{wrapfig}    %figures alignment
\usepackage{subcaption}     %for subfigures
%\usepackage{url}        %clickable links
\usepackage{longtable}  %multiple page tables

\usepackage{amsfonts} %per simboli insiemi numerici
\usepackage{amsmath} %per coefficienti binomiali
\usepackage{amsthm} %per teoremi non numerati
\usepackage{amssymb} %per simbolo insieme vuoto \varnothing

\usepackage{fancyhdr}
\pagestyle{fancy}

\usepackage{multirow} %per rowspan

\usepackage{tikz}
\usetikzlibrary{trees}

\usepackage{amsthm} %per teoremi non numerati

\newtheorem*{theorem}{Teorema}

\newtheorem*{definition}{DEF}

\newtheorem*{law}{Legge}

\newtheorem{ex}{Esempio}[section]

\newtheorem{axiom}{Assioma}

\newtheorem{corollary}{Corollario}

\title{Coniche}

\begin{document}
\lhead{Coniche}
    \chead{}
    \rhead{Davide Borra - 5LA}
\section*{Parabola}
\begin{table}[h]
    \centering
    \begin{tabular}{|m{0.16\textwidth}|m{0.37\textwidth}|m{0.37\textwidth}|}
        \hline
        & Asse parallelo all'asse $y$ & Asse parallelo all'asse $x$\\ \hline\hline
        Equazione & \[y=ax^2+bx+c\] & \[x=ay^2+by+c\]\\ \hline
        Concavità & \multicolumn{2}{m{0.74\textwidth}|}{\[\left\{\begin{array}{ll}
            \text{verso l'alto} & \text{se }a>0\\
            \text{verso il basso} & \text{se }a<0 
        \end{array}\right.\]}\\\hline
        Vertice & \[V\left(-\frac{b}{2a};-\frac{\Delta}{4a}\right)\] & \[V\left(-\frac{\Delta}{4a};-\frac{b}{2a}\right)\]\\\hline
        Asse & \[x=-\frac{b}{2a}\] & \[y=-\frac{b}{2a}\]\\\hline
        Fuoco &\[F\left(-\frac{b}{2a};\frac{1-\Delta}{4a}\right)\] & \[F\left(\frac{1-\Delta}{4a};-\frac{b}{2a}\right)\]\\\hline
        Direttrice & \[y=\frac{-1+\Delta}{4a}\] & \[x=\frac{-1+\Delta}{4a}\]\\\hline
        Sdoppiamento & \[\frac{y+y_P}{2}=axx_P+b\frac{x+x_P}{2}+c\] & \[\frac{x+x_P}{2}=ayy_P+b\frac{y+y_P}{2}+c\]\\\hline
    \end{tabular}
\end{table}
L'area del segmento parabolico compreso tra una retta $r$ secante alla parabola e che la interseca nei punti $A$ e $B$ è i $\frac{2}{3}$ dell'area del rettangolo $ABB'A'$, dove $A'$ e $B'$ sono le proiezioni dei punti $A$ e $B$ sulla retta $r'$ parallela a $r$ e tangente alla parabola. 
\section*{Circonferenza}
\begin{table}[h]
    \centering
    \begin{tabular}{|m{0.16\textwidth}|m{0.74\textwidth}|}
        \hline
        Equazione & \[(x-x_C)^2+(y-y_C)^2=r^2\] \[x^2+y^2+ax+by+c=0\] \\\hline
        Centro & \[C\left(-\frac{a}{2};-\frac{b}{2}\right)\] \\ \hline
        Raggio & \[r=\sqrt{\frac{a^2}{4}+\frac{b^2}{4}-c}\] \\ \hline
        Sdoppiamento & \[xx_P+yy_P+a\frac{x+x_P}{2}+b\frac{y+y_P}{2}+c=0\] \\ \hline
    \end{tabular}
\end{table}
\newpage
\section*{Ellisse}
    \begin{table}[h]
    \centering
    \begin{tabular}{|m{0.16\textwidth}|m{0.37\textwidth}|m{0.37\textwidth}|}
        \hline
        & Fuochi paralleli all'asse $x$ ($a>b$) & Fuochi paralleli all'asse $y$ ($a<b$)\\ \hline\hline
        Equazione & \multicolumn{2}{m{0.74\textwidth}|}{\[\frac{(x-x_C)^2}{a^2}+\frac{(y-y_C)^2}{b^2}=1\] \[Ax^2+By^2+Cx+Dy+E=0\]}\\ \hline
        Centro & \multicolumn{2}{m{0.74\textwidth}|}{\[C\left(x_C,y_C\right)\]}\\\hline
        Semiasse maggiore & \[a\] & \[b\]\\\hline
        Semiasse minore & \[b\] & \[a\]\\\hline
        Semidistanza focale & \[c=\sqrt{a^2-b^2}\] & \[c=\sqrt{b^2-a^2}\]\\ \hline
        Eccentricità & \[e=\frac{c}{a}\] & \[e=\frac{c}{b}\] \\ \hline
        Fuochi &\[F\left(\pm c+x_C;y_C\right)\] & \[F\left(x_C;\pm c + y_C\right)\]\\\hline
        Vertici & \multicolumn{2}{m{0.74\textwidth}|}{\[A\left(\pm a + x_C;y_C\right) ~~~~~B\left(x_C;\pm b +y_C\right)\]}\\\hline
        Sdoppiamento & \multicolumn{2}{m{0.74\textwidth}|}{\[Axx_P+Byy_P+C\frac{x+x_P}{2}+D\frac{y+y_P}{2}+E=0\]} \\ \hline
    \end{tabular}
\end{table}
\newpage
\section*{Iperbole}
        \begin{table}[h]
    \centering
    \begin{tabular}{|m{0.16\textwidth}|m{0.37\textwidth}|m{0.37\textwidth}|}
        \hline
        & Fuochi paralleli all'asse $x$ & Fuochi paralleli all'asse $y$\\ \hline\hline
        Equazione & \[\frac{(x-x_C)^2}{a^2}-\frac{(y-y_C)^2}{b^2}=1\]& \[\frac{(x-x_C)^2}{a^2}-\frac{(y-y_C)^2}{b^2}=-1\]\\ \hline 
        Equazione generica& \multicolumn{2}{m{0.74\textwidth}|}{
        \[Ax^2+By^2+Cx+Dy+E=0\]}\\ \hline
        Centro & \multicolumn{2}{m{0.74\textwidth}|}{\[C\left(x_C,y_C\right)\]}\\\hline
        Semiasse maggiore & \[a\] & \[b\]\\\hline
        Semiasse minore & \[b\] & \[a\]\\\hline
        Asintoti & \multicolumn{2}{m{0.74\textwidth}|}{\[y=\pm \frac{b}{a}(x-x_C)+y_C\]} \\ \hline
        Semidistanza focale & \multicolumn{2}{m{0.74\textwidth}|}{\[c=\sqrt{a^2+b^2}\]}\\ \hline
        Eccentricità & \[e=\frac{c}{a}\] & \[e=\frac{c}{b}\] \\ \hline
        Fuochi &\[F\left(\pm c+x_C;y_C\right)\] & \[F\left(x_C;\pm c + y_C\right)\]\\\hline
        Vertici & \multicolumn{2}{m{0.74\textwidth}|}{\[A\left(\pm a + x_C;y_C\right) ~~~~~B\left(x_C;\pm b +y_C\right)\]}\\\hline
        Sdoppiamento & \multicolumn{2}{m{0.74\textwidth}|}{\[Axx_P+Byy_P+C\frac{x+x_P}{2}+D\frac{y+y_P}{2}+E=0\]} \\ \hline
    \end{tabular}
\end{table}
\newpage
\subsection*{Iperbole equilatera riferita ai propri asintoti e funzione omografica}
        \begin{table}[h]
    \centering
    \begin{tabular}{|m{0.16\textwidth}|m{0.37\textwidth}|m{0.37\textwidth}|}
        \hline
        & $k>0$ & $k<0$\\ \hline\hline
        Equazione & \multicolumn{2}{m{0.74\textwidth}|}{
        \[y=\frac{ax+b}{cx+d}~~~~~\text{con }c\neq 0 \land ad-bc\neq 0\]}\\ \hline
        Centro & \multicolumn{2}{m{0.74\textwidth}|}{\[C\left(-\frac{d}{c};\frac{a}{c}\right)\]}\\\hline
        Costante di proporzionalità & \multicolumn{2}{m{0.74\textwidth}|}{\[k=\frac{bc-ad}{c^2}\]}\\\hline
        Semiassi & \multicolumn{2}{m{0.74\textwidth}|}{\[a=\sqrt{2|k|}\]}\\ \hline
        Asintoti & \multicolumn{2}{m{0.74\textwidth}|}{\[x=-\frac{d}{c}~~~~~y=\frac{a}{c}\]} \\ \hline
        Semidistanza focale & \multicolumn{2}{m{0.74\textwidth}|}{\[c=2\sqrt{|k|}\]}\\ \hline
        Eccentricità & \multicolumn{2}{m{0.74\textwidth}|}{\[e=\sqrt{2}\]} \\ \hline
        Fuochi &\[F\left(\pm a+x_C;\pm a+y_C\right)\] & \[F\left(\pm a + x_C; \mp a + y_C\right)\]\\\hline
        Vertici & \[A\left(\pm \sqrt{|k|} + x_C;\pm\sqrt{|k|}+y_C\right)\] & \[A\left(\pm \sqrt{|k|} + x_C;\mp\sqrt{|k|}+y_C\right)\]\\\hline
    \end{tabular}
\end{table}
\section*{Classificazione di coniche nella forma $Ax^2+Bxy+Cy^2+Dx+Ey+F=0$}
\subsection*{Metodo 1: Eccentricità}
\begin{definition}
Si dice conica il luogo dei punti P del piano tali che sia costante il rapporto $e$ (eccentricità della curva) tra le distanze da un punto $F$ detto fuoco e una retta $d$ detta direttrice:
\end{definition}
\[\frac{\overline{PF}}{d(P, d)}=e\]
L'eccentricità permette di classificare le coniche come segue:
\begin{itemize}
    \item $e>1$: iperbole (2 direttrici)
    \begin{itemize}
        \item $e=\sqrt{2}$: iperbole equilatera
    \end{itemize}
    \item $e=1$: parabola
    \item $0<e<1$: ellisse
    \item $e=0$: circonferenza
\end{itemize}
\subsection*{Metodo 2: Termine rettangolare} 
                \tikzstyle{level 1}=[level distance=1cm, sibling distance=4.5cm]
                \tikzstyle{level 2}=[level distance=2.5cm, sibling distance=2cm]
                \tikzstyle{level 3}=[level distance=3.5cm, sibling distance=0.9cm]
                
                \tikzstyle{bag} = [text width=1.5 cm, text centered]
                \tikzstyle{end} = [text width=9 cm, anchor = west]
                
                \begin{tikzpicture}[grow=right]
                \node[bag] {}
                    child {
                        node[bag] {$B=0$}        
                            child {
                                node[bag]{$A\neq 0$ $\lor$ $C\neq 0$}
                                    child {
                                        node[end]{Altra conica ruotata}
                                    }
                            }
                            child {
                                node[bag]{$A=0$ $\land$ $C=0$}
                                    child {
                                        node[end]{Omografica}
                                    }
                            }
                    }
                    child {
                        node[bag] {$B\neq0$}        
                            child {
                                node[bag]{$AC>0$}
                                    child {
                                        node[end]{Ellisse se $s>0$ e concorde con $A$ e $C$ (cfr se $A=C$)}
                                    }child {
                                        node[end]{Ellisse degenere in punto se $s=0$}
                                    }
                            }
                            child {
                                node[bag]{$AC<0$}
                                    child {
                                        node[end]{Iperbole se $s=\frac{D^2}{4A}+\frac{E^2}{4C}-F\neq0$}
                                    }child {
                                        node[end]{Iperbole degenere in rette se $s=0$}
                                    }
                            }
                            child {
                                node[bag]{$AC=0$}
                                    child {
                                        node[end]{Parabola}
                                    }child {
                                        node[end]{Parabola degenere in rette ($A=D=0 \lor C=E=0$)}
                                }
                            }
                    };
                \end{tikzpicture}
\subsection*{Metodo 3: Matrice associata}
Si costruisce la matrice \[\renewcommand{\arraystretch}{2}A=\left[\begin{array}{ccc}
    A & \dfrac{B}{2} & \dfrac{D}{2}\\
    \dfrac{B}{2} & C & \dfrac{E}{2} \\
    \dfrac{D}{2} & \dfrac{E}{2} & F
\end{array}\right]\]
Il cui minore principale di ordine 2 è la matrice 
\[\renewcommand{\arraystretch}{2}M=\left[\begin{array}{cc}
    A & \dfrac{B}{2} \\
    \dfrac{B}{2} & C
\end{array}\right]\]
\begin{itemize}
    \item $detM>0$: ellisse
        \begin{itemize}
            \item $detA(A+C)>0$ ellisse complessa
            \item $detA(A+C)<0$ ellisse reale
            \item $detA=0$ ellisse degenere
            \item $B=0 \land A=C$ circonferenza (reale o complessa come sopra)
        \end{itemize}
    \item $detM=0$: parabola
        \begin{itemize}
            \item $detA(A+C)>0$ parabola complessa
            \item $detA(A+C)<0$ parabola reale
            \item $detA=0$ parabola degenere
        \end{itemize}
    \item $detM<0$: iperbole
    \begin{itemize}
            \item $detA(A+C)>0$ iperbole complessa
            \item $detA(A+C)<0$ iperbole reale
            \item $detA=0$ iperbole degenere
            \item $A+C=0$ iperbole equilatera (reale o complessa come sopra)
        \end{itemize}
\end{itemize}
\end{document}
