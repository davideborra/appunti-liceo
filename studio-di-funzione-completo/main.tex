\documentclass{article}     %type of document
\usepackage[utf8]{inputenc} %for text encoding
\usepackage[italian]{babel} %for language elements
\usepackage{graphicx}   %for adding figures
\usepackage[a4paper, portrait, margin=2cm]{geometry}   %paper shape
\usepackage{array}      %table management
\usepackage{wrapfig}    %figures alignment
\usepackage{subcaption}     %for subfigures
%\usepackage{url}        %clickable links
\usepackage{longtable}  %multiple page tables
\usepackage{multirow}
\usepackage{tikz}
%\usetikzlibrary{trees}
\usetikzlibrary{calc}

%geogebra
\usepackage{pgfplots}
\pgfplotsset{compat=1.15}
\usepackage{mathrsfs}
\usetikzlibrary{arrows}

\usepackage{amsfonts} %per simboli insiemi numerici
\usepackage{amsmath} %per coefficienti binomiali
\usepackage{amsthm} %per teoremi non numerati
\usepackage{amssymb} %per simbolo insieme vuoto \varnothing

%intestazione
\usepackage{fancyhdr}
\pagestyle{fancy}
\newcommand\parallelo{/\!/} %simbolo parallelismo
\newcommand\tg{\text{\textnormal{tg\,}}} %tangente
\newcommand\cotg{\text{\textnormal{cotg\,}}} %cotangente
\newcommand\arctg{\text{\textnormal{arctg\,}}} %arcotangente
\newcommand\arccotg{\text{\textnormal{arccotg\,}}} %arcocotangente

\newcommand\N{\mathbb{N}}
\newcommand\Z{\mathbb{Z}}
\newcommand\Q{\mathbb{Q}}
\newcommand\R{\mathbb{R}}
\newcommand\C{\mathbb{C}}


\pgfplotsset{compat=1.15}
\usepackage{mathrsfs}
\usetikzlibrary{arrows}

\usepackage{standalone} %external matlab tikz files

\newtheorem*{theorem}{Teorema}

\newtheorem*{definition}{DEF}

\newtheorem*{law}{Legge}

\newtheorem{ex}{Esempio}[section]
\newtheorem*{ex*}{Esempio}

\newtheorem{axiom}{Assioma}

\newtheorem{corollary}{Corollario}


\title{Studio di funzione completo}
\author{Davide Borra - 5LA}
\date{A.S. 2021-2022}

\begin{document}
\lhead{Studio di funzione completo}
\chead{}
\rhead{Davide Borra - 5LA}
\section{Studio di funzione completo}
    \subsection{Classificazione}
    \begin{tabular}{|c|c|c|c|}
        \hline
        \multirow{2}{4em}{Funzione} & algebrica\ & razionale & intera\\ \cline{2-4}
         & trascendente & irrazionale & fratta \\ \hline
    \end{tabular}
    \subsection{Dominio}
    \begin{itemize}
        \item Polinomiale : $\R$
        \item Fratte: denominatore $\neq 0$
        \item Irrazionali pari: radicando$\geq0$
        \item Irrazionali dispari: $\R$
        \item Logaritmi: argomento$>0$
        \item Esponenziali: $\R$
        \item Seno, coseno, arcotangente, arcocotangente: $\R$
        \item Tangente: $\R -{\frac{\pi}{2}+k\pi}\text{, con } k\in\Z$
        \item Cotangente: $\R -{k\pi}\text{, con } k\in\Z$
        \item Arcoseno, arcocoseno: $[-1;1]$
    \end{itemize}
    \subsection{Simmetrie}
    \textbf{CN:} $\forall x \in D, -x \in D$
    \begin{itemize}
        \item pari se $f(-x)=f(x)$
        \item dispari se $f(-x)=-f(x)$
    \end{itemize}
    \subsection{Intersezioni con gli assi cartesiani}
    \[f(x) \cap \text{asse~}x : \left\{\begin{array}{l}
         y=f(x)  \\
         y=0 
    \end{array}\right.~~~~~~~~ 
    f(x) \cap \text{asse~}y : \left\{\begin{array}{l}
         y=f(x)  \\
         x=0 
    \end{array}\right.\] 
    \subsection{Studio del segno}
    Risolvere la disequazione $f(x)>0$
    \subsection{Limiti, asintoti e discontinuità}
    \begin{tabular}{|m{0.3\textwidth}|m{0.6\textwidth}|}
        \hline
        Asintoto verticale&  \[\lim_{x\rightarrow x_0} f(x) = \infty\] \\\hline
        Asintoto orizzontale & \[\lim_{x\rightarrow \infty} f(x) = l\]\\ \hline
        Asintoto obliquo & \[\textnormal{CN: }\lim_{x\rightarrow\infty}f(x) = \infty\]
        \[m=\lim_{x\rightarrow\infty}\frac{f(x)}{x}\]
        \[q=\lim_{x\rightarrow\infty}f(x)-mx\] \\ \hline
    \end{tabular}\\
    \\
    \textbf{NB.:} Una funzione può avere anche infiniti asintoti verticali, ma al massimo due tra asintoti orizzontali e asintoti obliqui (uno destro e uno sinistro).\\    \begin{tabular}{|m{0.3\textwidth}|m{0.6\textwidth}|}
        \hline
        Prima specie &  \[\lim_{x\rightarrow x_0^-}f(x)=l_1~~~~\lim_{x\rightarrow x_0^+}f(x)=l_2\] \[l_1\neq l_2~~~~salto=|l_1-l_2|\]\\ \hline
        Seconda specie &  \[\lim_{x\rightarrow x_0^\pm}f(x)=\infty ~~~~\lor~~~~ \lim_{x\rightarrow x_0^\pm}f(x)=\nexists\] \\\hline
        Terza specie & \[\lim_{x\rightarrow x_0^-}f(x)=\lim_{x\rightarrow x_0^+}f(x)=l\]\[f(x_0)\neq l ~~~~ \lor ~~~~ f(x_0)=\nexists\]\\ \hline
    \end{tabular}
    \subsection{Derivata prima}
    Le soluzioni dell'equazione \[f'(x) = 0\] identificano la presenza di \begin{itemize}
        \item massimi relativi
        \item minimi relativi
        \item flessi a tangente orizzontale
    \end{itemize}
    Per distinguerli è necessario studiare il segno della derivata:
    \begin{figure}[h]
        \centering
        \begin{subfigure}{0.48\textwidth}
            \centering
            \begin{tikzpicture}
                \draw[->](0,0)--(3,0);
                \draw(1.5,0) node[above]{$x_0$}--(1.5,-2);
                \node at (-0.50, -0.60){$f(x)$};
                \node at (-0.50, -1.50){$f'(x)$};
                \node at (0.75,-0.60) {$+$};
                \node at (2.25,-0.60) {$-$};
                \node at (1.5, -0.60) {$\circ$};
                \draw[->] (0.4,-1.85)--(1.1,-1.15); %+ sx
                \draw[->] (1.9,-1.15)--(2.6,-1.85); %- dx
            \end{tikzpicture}
            \caption{Minimo relativo}
        \end{subfigure}
        \begin{subfigure}{0.48\textwidth}
            \centering
            \begin{tikzpicture}
                \draw[->](0,0)--(3,0);
                \draw(1.5,0) node[above]{$x_0$}--(1.5,-2);
                \node at (-0.50, -0.60){$f(x)$};
                \node at (-0.50, -1.50){$f'(x)$};
                \node at (0.75,-0.60) {$-$};
                \node at (2.25,-0.60) {$+$};
                \node at (1.5, -0.60) {$\circ$};
                \draw[->] (0.4,-1.15)--(1.1,-1.85); %- sx
                \draw[->] (1.9,-1.85)--(2.6,-1.15); %+ dx
            \end{tikzpicture}
            \caption{Massimo relativo}
        \end{subfigure}
        \begin{subfigure}{0.48\textwidth}
            \centering
            \begin{tikzpicture}
                \draw[->](0,0)--(3,0);
                \draw(1.5,0) node[above]{$x_0$}--(1.5,-2);
                \node at (-0.50, -0.60){$f(x)$};
                \node at (-0.50, -1.50){$f'(x)$};
                \node at (0.75,-0.60) {$+$};
                \node at (2.25,-0.60) {$+$};
                \node at (1.5, -0.60) {$\circ$};
                %\draw[->] (0.4,-1.15)--(1.1,-1.85); %- sx
                \draw[->] (0.4,-1.85)--(1.1,-1.15); %+ sx
                %\draw[->] (1.9,-1.15)--(2.6,-1.85); %- dx
                \draw[->] (1.9,-1.85)--(2.6,-1.15); %+ dx
            \end{tikzpicture}
            \caption{Flesso a tangente orizzontale ascendente}
        \end{subfigure}
        \begin{subfigure}{0.48\textwidth}
            \centering
            \begin{tikzpicture}
                \draw[->](0,0)--(3,0);
                \draw(1.5,0) node[above]{$x_0$}--(1.5,-2);
                \node at (-0.50, -0.60){$f(x)$};
                \node at (-0.50, -1.50){$f'(x)$};
                \node at (0.75,-0.60) {$-$};
                \node at (2.25,-0.60) {$-$};
                \node at (1.5, -0.60) {$\circ$};
                \draw[->] (0.4,-1.15)--(1.1,-1.85); %- sx
                \draw[->] (1.9,-1.15)--(2.6,-1.85); %- dx
            \end{tikzpicture}
            \caption{Flesso a tangente verticale discendente}
        \end{subfigure}
    \end{figure}
    Lo studio della derivata prima fornisce anche informazioni circa la monotonia della funzione.
    \subsection{Derivata seconda}
    Le soluzioni dell'equazione \[f''(x)=0\] permettono di identificare i punti di flesso. A differenza della derivata prima permette di ottenere informazioni circa la presenza di flessi a tangente obliqua, per cui è necessario escludere tutte le soluzioni già analizzate in precedenza. 
    La derivata seconda fornisce inoltre informazioni circa la concavità della funzione: verso l'alto quando la derivata seconda è positiva e verso il basso quando la derivata seconda è negativa. La funzione inverte la propria concavità in corrispondenza dei punti di flesso.
\end{document}