\documentclass{article}     %type of document
\usepackage[utf8]{inputenc} %for text encoding
\usepackage[italian]{babel} %for language elements
\usepackage{graphicx}   %for adding figures
\usepackage[a4paper, portrait, margin=2cm]{geometry}   %paper shape
\usepackage{array}      %table management
\usepackage{wrapfig}    %figures alignment
%\usepackage{url}        %clickable links
%\usepackage{longtable}  %multiple page tables
%trees
\usepackage{tikz}
\usetikzlibrary{trees}
\usetikzlibrary{calc}
%geogebra
\usepackage{pgfplots}
\pgfplotsset{compat=1.15}
\usepackage{mathrsfs}
\usetikzlibrary{arrows}
\usepackage{amsfonts} %per simboli insiemi numerici
\usepackage{amsmath} %per coefficienti binomiali
\usepackage{amsthm} %per teoremi non numerati
\usepackage{amssymb} %per simbolo insieme vuoto \varnothing
%intestazione
\usepackage{fancyhdr}
\pagestyle{fancy}
\usepackage{subcaption}     %for subfigures
\newcommand\parallelo{/\!/} %simbolo parallelismo

\pgfplotsset{compat=1.15}
\usepackage{mathrsfs}
\usetikzlibrary{arrows}

\usepackage{standalone} %external matlab tikz files

\newtheorem*{theorem}{Teorema}

\newtheorem*{definition}{DEF}

\newtheorem*{law}{Legge}

\newtheorem{ex}{Esempio}[section]

\newtheorem{axiom}{Assioma}

\newtheorem{corollary}{Corollario}


\title{Geometria analitica nello spazio}
\author{Davide Borra - 5LA}
\date{A.S. 2021-2022}

\begin{document}
    \begin{titlepage}
    \maketitle
    \tableofcontents
    \end{titlepage}
    
    \lhead{Geometria analitica nello spazio}
    \chead{}
    \rhead{Davide Borra - 5LA}
    \section{Le coordinate cartesiane}
        Un punto nel piano è identificato univocamente da una coppia di numeri reali, corrispondenti alla distanza delle sue proiezioni sugli assi dall'intersezione degli stessi. Nello spazio i punti vengono individuati nello stesso modo, ma è necessario introdurre un'ulteriore asse (e di conseguenza un'ulteriore coordinata). Un punto nello spazio è individuato quindi tramite una terna ordinata di numeri reali $(x,y,z)$, chiamati rispettivamente \textit{ascissa}, \textit{ordinata} e \textit{quota}.
        \begin{center}
            \begin{figure}[h]
                \begin{subfigure}{0.30\textwidth}
                    \begin{center}
                        \begin{tikzpicture}[scale=0.8]
                            \draw[help lines, color=white];
                            \draw[->,thin] (-1,0)--(4,0) node[right]{$x$};
                            \draw[->,thin] (0,-1)--(0,4) node[left]{$y$};
                            
                            \node[rotate=90] at (-1, 2) {~~Asse delle ordinate};
                            \node[rotate=0] at (2, -1) {~~Asse delle ascisse};
    
                            \node at (2,3) {\textbullet};
                            \node at (2.3, 3.3) {$A(3,2)$};
                            
                            \draw[dashed] (2,3)--(2,0) node[below]{2};
                            \draw[dashed] (2,3)--(0,3) node[left]{3};
                        \end{tikzpicture}
                    \end{center}
                    \caption{Nel piano}
                \end{subfigure}
                \begin{subfigure}{0.30\textwidth}
                    \begin{center}
                        \begin{tikzpicture}[scale=0.8]
                            \draw[help lines, color=white];
                            \draw[->,thin] (0,0,0)--(3,0,0) node[right]{$x$};
                            \draw[->,thin] (0,0,0)--(0,3,0) node[left]{$z$};
                            \draw[->,thin] (0,0,0)--(0,0,4) node[left]{$y$};
        
                            \draw[dashed] (-0.6,0,0)--(0,0,0);
                            \draw[dashed] (0,-0.6,0)--(0,0,0);
                            \draw[dashed] (0,0,-1)--(0,0,0);
                        \end{tikzpicture}
        
                    \end{center}
                    \caption{Sistema sinistro}
                \end{subfigure}
                \begin{subfigure}{0.30\textwidth}
                    \begin{center}
                        \begin{tikzpicture}[scale=0.8]
                            \draw[help lines, color=white];
                            \draw[->,thin] (0,0,0)--(3,0,0) node[right]{$y$};
                            \draw[->,thin] (0,0,0)--(0,3,0) node[left]{$z$};
                            \draw[->,thin] (0,0,0)--(0,0,4) node[left]{$x$};
        
                            \draw[dashed] (-0.6,0,0)--(0,0,0);
                            \draw[dashed] (0,-0.6,0)--(0,0,0);
                            \draw[dashed] (0,0,-1)--(0,0,0);
        
                            \node at (2,2,3) {\textbullet};
                            \node at (2.7, 2.5, 3.3) {$A(3,2,3)$};
                            
                            \draw[dashed] (2,0,3)--(2,0,0) node[above]{2};
                            \draw[dashed] (2,2,3)--(0,2,0) node[left]{2};
                            \draw[dashed] (2,0,3)--(0,0,3) node[left]{3};
                            \draw[dashed] (2,2,3)--(2,0,3);
                        \end{tikzpicture}
                    \end{center}
                    \caption{Sistema destro (il più utilizzato)}
                \end{subfigure}
                \caption{Coordinate cartesiane}
            \end{figure}    

        \end{center}
        
        \subsection{Quadranti e ottanti}
            Come il piano cartesiano è diviso in 4 quandranti (Fig. \ref{fig:quadranti}) dagli assi cartesiani, lo spazio cartesiano è diviso in 8 ottanti (Fig. \ref{fig:ottanti}).

            \begin{center}
                \begin{figure}[h]
                    \begin{subfigure}{0.49\textwidth}
                        \begin{center}
                            \begin{tikzpicture}
                                \draw[help lines, color=white];
                                    \draw[->,thin] (-2,0)--(2,0) node[right]{$x$};
                                    \draw[->,thin] (0,-2)--(0,2) node[left]{$y$};
            
                                    \node at (1,1) {1};
                                    \node at (-1,1) {2};
                                    \node at (-1,-1) {3};
                                    \node at (1,-1) {4};
                            \end{tikzpicture}
                        \end{center}
                        \caption{Quadranti}
                        \label{fig:quadranti}
                    \end{subfigure}
                    \begin{subfigure}{0.49\textwidth}
                        \begin{center}
                            \begin{tikzpicture}
                                \draw[help lines, color=white];
                                \draw[->,thin] (0,0,0)--(2,0,0) node[right]{$y$};
                                \draw[->,thin] (0,0,0)--(0,2,0) node[left]{$z$};
                                \draw[->,thin] (0,0,0)--(0,0,3) node[left]{$x$};

                                \draw[dashed] (-2,0,0)--(0,0,0);
                                \draw[dashed] (0,-2,0)--(0,0,0);
                                \draw[dashed] (0,0,-3)--(0,0,0);
            
                                \node at (1,1,1) {1};
                                \node at (-1,1,1) {2};
                                \node at (-1,-1,1) {5};
                                \node at (1,-1,1) {6};
                                \node at (1,1,-1) {4};
                                \node at (-1,1,-1) {3};
                                \node at (-1,-1,-1) {7};
                                \node at (1,-1,-1) {8};
                            \end{tikzpicture}
                        \end{center}
                        \caption{Ottanti}
                        \label{fig:ottanti}
                    \end{subfigure}
                    \caption{Suddivisione del piano e dello spazio cartesiano}
                \end{figure}
            \end{center}
        
        \subsection{Piani coordinati}
            Esistono alcuni piani particolari, detti coorinati, in cui una delle variabili assume il valore 0. Essi sono:
            \begin{wrapfigure}{r}{0.2\textwidth}
                \begin{center}
                    \begin{tikzpicture}[scale=0.5]
                        \draw[help lines, color=white];
                        \draw[->,thin] (0,0,0)--(2,0,0) node[right]{$y$};
                        \draw[->,thin] (0,0,0)--(0,2,0) node[left]{$z$};
                        \draw[->,thin] (0,0,0)--(0,0,3) node[left]{$x$};
    
                        \draw[dashed] (-2,0,0)--(0,0,0);
                        \draw[dashed] (0,-2,0)--(0,0,0);
                        \draw[dashed] (0,0,-3)--(0,0,0);
    
                        \draw[fill=gray, color=gray, opacity=0.5] (-2,0,-3) -- (-2,0,3) -- (2,0,3) -- (2,0,-3) -- cycle;
    
                    \end{tikzpicture}
                \end{center}
                \caption{Il piano\\ coordinato $xy$}
                \label{fig:pianoxy}
            \end{wrapfigure}
            \begin{itemize}
                \item il piano $xy$ quando $z=0$ (Fig. \ref{fig:pianoxy}). È costituito da punti di coordinate $(x,y,0)$.
                \item il piano $xz$ quando $y=0$. È costituito da punti di coordinate $(x,0,z)$.
                \item il piano $zy$ quando $x=0$. È costituito da punti di coordinate $(0,y,z)$.
            \end{itemize}
            
            \subsection{Assi coordinati}
            Analogamente i tre assi cartesiani sono detti assi coordinati: 
            \[\text{Asse }x:\left\{\begin{array}{c}
                y=0\\
                z=0
            \end{array}\right.
            \text{~~~Asse }y:\left\{\begin{array}{c}
                x=0\\
                z=0
            \end{array}\right.
            \text{~~~Asse }z:\left\{\begin{array}{c}
                x=0\\
                y=0
            \end{array}\right.
            \]

        \subsection{Distanza tra due punti}
            Consideriamo due punti $A(x_A,y_A,z_A)$ e $B(x_B,y_B,z_B)$, la distanza tra essi è \[ \overline{AB}=\sqrt{(x_A-x_B)^2+(y_A-y_B)^2+(z_A-z_B)^2}\]
            La formula si dimostra facilmente tramite l'utilizzo del Teorema di Pitagora.

        \subsection{Punti notevoli}
            \paragraph*{Ricorda:}
                \begin{tabular}[h]{|c|c|}
                    \hline
                    Punto nottevole & Segmenti ceviani corrispondenti\\
                    \hline \hline
                    Baricentro & Mediane\\ \hline
                    Circocentro & Assi\\ \hline
                    Ortocentro & Altezze\\ \hline
                    Incentro & Bisettrici \\ \hline
                \end{tabular}
            \subsubsection{Punto medio}
                Dati due punti $A(x_A,y_A,z_A)$ e $B(x_B,y_B,z_B)$, le coordinate del punto medio $M(x_M, y_M, z_M)$ sono date dalla media delle coordinate dei due punti.
                \[x_M=\frac{x_A+x_B}{2}~~~~~~y_M=\frac{y_A+y_B}{2}~~~~~~z_M=\frac{z_A+z_B}{2}\]
            \subsubsection{Baricentro}
                Dato un triangolo $ABC$ di vertici $A(x_A,y_A,z_A)$, $B(x_B,y_B,z_B)$, $C(x_C,y_C,z_C)$, le coordinate del baricentro $G(x_G,y_G,z_G)$ sono date dalla media delle coordinate dei tre punti:
                \[x_G=\frac{x_A+x_B+x_C}{3}~~~~~~y_G=\frac{y_A+y_B+y_C}{3}~~~~~~z_G=\frac{z_A+z_B+z_C}{3}\]

    \section{Vettori}
        \subsection{Vettori nel piano}
            \begin{wrapfigure}{r}{0.30\textwidth}
                \begin{tikzpicture}[line cap=round,line join=round,>=triangle 45,x=1.0cm,y=1.0cm]
                    \begin{axis}[
                    x=1.0cm,y=1.0cm,
                    axis lines=middle,
                    xmin=-1.5,
                    xmax=3.5,
                    ymin=-1.0,
                    ymax=3.5,
                    xtick={1.0},
                    ytick={1.0},
                    xlabel=$x$,
                    ylabel=$y$,
                    ]
                    \draw[->] (0,0) to (2,2.3) node[above]{$P(a,b)$};
                    \draw[dashed] (2,2.3) to (0,2.3) node[left]{$B(0,b)$};
                    \draw[dashed] (2,2.3) to (2,0) node[below]{$A(a,0)$};
        
                    \draw[->, thick] (0,0) to (1,0);
                    \node at (0.5,-0.3){$\overrightarrow{i}$};
                    \draw[->, thick] (0,0) to (0,1);
                    \node at (-0.2,0.3){$\overrightarrow{j}$};
                    \end{axis}
                    \end{tikzpicture}
                    \caption{Vettori nel piano}
            \end{wrapfigure}

            Si definisce \textbf{vettore} un segmento orientato che ha un estremo nell'origine. Esso è individuato da una terna ordinata di coordinate, dette componenti, che identificano l'altro estremo del vettore, verso cui è orientato. \\
            Si dice \textbf{versore} un vettore di lunghezza unitaria parallelo ad un asse cartesiano. I versori nel piano sono $\overrightarrow{i}(1,0)$ e $\overrightarrow{j}(1,0)$.\\
            I vettori possono essere rappresentati due in diversi modi:
            \begin{itemize}
                \item Tramite una terna di coordinate: $\overrightarrow{v}(v_x, v_y, v_z)$
                \item Tramite i versori: $\overrightarrow{v}=v_x\overrightarrow{i}+v_y\overrightarrow{j}$
            \end{itemize}
            I vettori $\overrightarrow{OA}$ e $\overrightarrow{OB}$ soo detti vettori componenti del vettore $\overrightarrow{OP}$.
            Il modulo di un vettore è la lunghezza del segmento orentato, per cui si calcola attraverso la formula per la distanza tra due punti:
            \[|\overrightarrow{v}|=v=\sqrt{v_x^2+v_y^2}\]
        \subsection{Vettori nello spazio}
            Per passare dal piano allo spazio è necessario introdurre una nuova coordinata, $z$. Di conseguenza è necessario introdurre anche una nuova componente dei vettori e un nuovo versore. I versori nello spazio saranno quindi:
            \begin{itemize}
                \item $\overrightarrow{i}(1,0,0)$, associato all'asse $x$
                \item $\overrightarrow{j}(0,1,0)$, associato all'asse $y$
                \item $\overrightarrow{k}(0,0,1)$, associato all'asse $z$
            \end{itemize}
            I vettori saranno quindi individuati da una terna ordinata di numeri reali o da una somma di componenti:
            \[\overrightarrow{v}(v_x, v_y, v_z)\]
            \[\overrightarrow{v}=v_x\overrightarrow{i}+v_y\overrightarrow{j}+v_z\overrightarrow{k}\]

            Valgono ancora tutte le proprietà dei vettori nel piano:

            \renewcommand{\labelenumi}{\alph{enumi})}
            \begin{enumerate}
                \item il modulo del vettore $\overrightarrow{v}(v_x, v_y, v_z)$ è dato da \[|\overrightarrow{v}|=v=\sqrt{v_x^2+v_y^2+v_z^2}\]
                \item le componenti di un vettore $\overrightarrow{v}(v_x, v_y, v_z)=\overrightarrow{AB}$ con $A(x_A,y_A,z_A)$ e $B(x_B,y_B,z_B)$ sono 
                \[\begin{array}{c}
                    v_x=x_B-x_A\\
                    v_y=y_B-y_A\\
                    v_z=z_B-z_A
                \end{array}\]
                \item dati due vettori $\overrightarrow{v}(v_x, v_y, v_z)$ e $\overrightarrow{\mu}(\mu_x, \mu_y, \mu_z)$ 
                    \[\begin{array}{c}
                        \overrightarrow{v}+\overrightarrow{\mu}(v_x+\mu_x, v_y+\mu_y, v_z+\mu_z)\\ \\
                        \overrightarrow{v}-\overrightarrow{\mu}(v_x-\mu_x, v_y-\mu_y, v_z-\mu_z)\\ \\
                        k\overrightarrow{\mu}(k\mu_x, k\mu_y, k\mu_z)\\ \\
                        \text{prodotto scalare:~~~~} \overrightarrow{v}\cdot\overrightarrow{\mu}=v_x\mu_x + v_y\mu_y+v_z\mu_z
                    \end{array}\]
            \end{enumerate}
            \renewcommand{\labelenumi}{\arabic{enumi}.} %resets default list type
            
        \subsection{Vettori paralleli e perpendicolari}
            \begin{definition}
                Due vettori si dicono \textbf{paralleli} quando hanno, una volta traslati nell'origine, la stessa retta d'azione, ovvero giacciono sulla stessa retta (indipendentemente dal verso). Questo significa che un vettore è multiplo dell'altro. 
                
                \[\overrightarrow{v}\parallelo\overrightarrow{w}\Leftrightarrow \overrightarrow{v}=k\overrightarrow{w}\]
            \end{definition}
            
            Per determinare quando due vettori sono perpendicolari invece è possibile sfruttare una caratteristica del prodotto scalare: esso infatti assume valore nullo quando i due vettori considerati sono perpendicolari. Di conseguenza:
            \[\overrightarrow{v}\perp\overrightarrow{w}\Leftrightarrow \overrightarrow{v}\cdot\overrightarrow{w}=0\]

            \section{Piani}
            Dati una retta $r$ e un punto $P_0(x_0, y_0, z_0)$ qualsiasi, esiste uno e un solo piano passante per $P$ e perpendicolare a $r$. Il piano resta univocamente determinato se alla retta si sostituisce un vettore parallelo alla stessa, detto \textbf{vettore normale del piano} $\overrightarrow{n}(a,b,c)$. 
            \begin{figure}[h]
                \begin{subfigure}{0.49\textwidth}
                    \begin{center}
                        \begin{tikzpicture}[scale=0.8]
                            \draw[help lines, color=white];
                            \draw[thin] (0,0,0)--(0,2,0) node[left]{$r$};
            
                            \draw[dashed] (0,-2,0)--(0,0,0);
            
                            \node at (0,0,0){\textbullet};
                            \node at (1,0,-1){\textbullet};
                            \node at (1,0.5,-1){$P_0$};
            
                            \draw[fill=gray, color=gray, opacity=0.5] (-2,0,-3) -- (-2,0,3) -- (2,0,3) -- (2,0,-3) -- cycle;
            
                        \end{tikzpicture}
                    \end{center}
                    \caption{Retta perpendicolare e punto}
                \end{subfigure}
                \begin{subfigure}{0.49\textwidth}
                    \begin{center}
                        \begin{tikzpicture}[scale=0.8]
                            \draw[help lines, color=white];
                            \draw[->,thin] (0,0,0)--(0,1.5,0) node[right]{$\overrightarrow{n}$};

                            \node at (0,0,0){\textbullet};
                            \node at (-0.6,0,0){$P_0$};

                            \draw[fill=gray, color=gray, opacity=0.5] (-2,0,-3) -- (-2,0,3) -- (2,0,3) -- (2,0,-3) -- cycle;
            
                        \end{tikzpicture}
                    \end{center}
                    \caption{Vettore normale e punto}
                \end{subfigure}
                \caption{Metodi di individuazione di un piano}
                \label{fig:defpiano}
            \end{figure}

            \begin{wrapfigure}{r}{0.49\textwidth}
                \begin{center}
                    \begin{tikzpicture}[scale=0.8]
                        \draw[help lines, color=white];
                        \draw[->] (0,0,0)--(0,1.5,0) node[right]{$\overrightarrow{n}$};
                        \draw[->] (0,0,0)--(1,0,-1);
                            \node at (0.8,0,0.3) {$\overrightarrow{P_0P}$};

                        \draw (0,0.2,0)--(0.141,0.2,-0.141)--(0.141,0,-0.141);

                        \node at (0,0,0){\textbullet};
                            \node at (-0.6,0,0){$P_0$};
                        \node at (1,0,-1){$\mathbf{\cdot}$};
                            \node at (1,0.3,-1){$P$};

                        \draw[fill=gray, color=gray, opacity=0.5] (-2,0,-3) -- (-2,0,3) -- (2,0,3) -- (2,0,-3) -- cycle;
        
                    \end{tikzpicture}
                \end{center}
                \caption{Definizione del piano}
            \end{wrapfigure}
            Un punto $P(x,y,z)$ appastiene al piano $\pi$ se e solo se $\overrightarrow{P_0P}\perp \overrightarrow{n}$
            \[\overrightarrow{P_0P}\perp\overrightarrow{n}\]
            \[\overrightarrow{P_0P}\cdot\overrightarrow{n}=0\]
            \[P(x,y,z)~~~~~P_0(x_0,y_0,z_0)\]
            \[\overrightarrow{P_0P}(x-x_0,y-y_0,z-z_0)~~~~~\overrightarrow{n}(a,b,c)~~~~~~~~~~~~~~~~~~~~~~~~~~~~~~~~~~~~~~~~~~~~~~~~~~~~~~~~~~~~~~~~~~~~~~\]
            \[(a,b,c)(x-x_0,y-y_0,z-z_0)=0~~~~~~~~~~~~~~~~~~~~~~~~~~~~~~~~~~~~~~~~~~~~~~~~~~~~~~~~~~~~~~~~~~~~~~\]
            \[a(x-x_0)+b(y-y_0)+c(z-z_0)=0~~\rightarrow~~\text{equazione del piano passante per un punto dao vettore normale}\]
            \[ax+by+cz+(-ax_0-by_0-cz_0)=0)~~~~~~~~~~~~~~~~~~~~~~~~~~~~~~~~~~~~~~~~~~~~~~~~~~~~~~~~~~~~~~~~~~~~~~\]
            \[\mathbf{ax+by+cz+d}=0~~~~\rightarrow~~~~\text{equazione del piano in forma cartesiana}~~~~~~~~~~~~\]

            \subsection{Posizioni notevoli dei piani}
            \[\pi: ax+by+cz+d=0\]
            
            Se $d=0$ il piano passa per l'origine. 
            
            Se nell'equazione del piano almeno uno dei coefficienti $a$, $b$ o $c$ è nullo, si ottiene un piano che risulta parallelo all'asse corrispondente alla variabile mancante.

            \begin{center}
                \begin{figure}[h]
                    \begin{subfigure}{0.3\textwidth}
                        \begin{center}
                            \begin{tikzpicture}[scale=0.8]
                                \draw[help lines, color=white];
                                \draw[->,thin] (0,0,0)--(3,0,0) node[right]{$y$};
                                \draw[->,thin] (0,0,0)--(0,3,0) node[left]{$z$};
                                \draw[->,thin] (0,0,0)--(0,0,4) node[left]{$x$};
                            
                                \draw[fill=gray, color=gray, opacity=0.3] (0,0,0) -- (0,0,3.7) -- (0,2.7,3.7) -- (0,2.7,0) -- cycle;    %piano xz
                                \draw[fill=gray, color=gray, opacity=0.3] (0,0,0) -- (0,2.7,0) -- (2.7,2.7,0) -- (2.7,0,0) -- cycle;    %piano yz
                            
                                \draw[fill=cyan, color=cyan, opacity=0.5] (2,0,-1) -- (2,2,-1) -- (-1,2,2) -- (-1,0,2) -- cycle;    %piano // z
                                \draw (1,-0.5,0)--(1,2.5,0);
                                    \draw[dashed] (1,-0.5,0)--(1,-1.1,0);
                                    \draw[dashed] (1,3,0)--(1,2.5,0);
                                \draw (0,-0.5,1)--(0,2.5,1);
                                    \draw[dashed] (0,-0.5,1)--(0,-1.1,1);
                                    \draw[dashed] (0,3,1)--(0,2.5,1);
                            \end{tikzpicture}
                        \end{center}
                        \caption{$ax+by+d=0$\\Piano parallelo all'asse $z$\\Interseca i piani $xz$ e $yz$ lungo\\ rette parallele all'asse $z$.}
                    \end{subfigure}
                    \begin{subfigure}{0.3\textwidth}
                        \begin{center}
                            \begin{tikzpicture}[scale=0.8]
                                \draw[help lines, color=white];
                                \draw[->,thin] (0,0,0)--(3,0,0) node[right]{$y$};
                                \draw[->,thin] (0,0,0)--(0,3,0) node[left]{$z$};
                                \draw[->,thin] (0,0,0)--(0,0,4) node[left]{$x$};
                            
                                \draw[fill=gray, color=gray, opacity=0.3] (0,0,0) -- (0,0,3.7) -- (2.7,0,3.7) -- (2.7,0,0) -- cycle;    %piano xy
                                \draw[fill=gray, color=gray, opacity=0.3] (0,0,0) -- (0,2.7,0) -- (2.7,2.7,0) -- (2.7,0,0) -- cycle;    %piano yz
                            
                                \draw[fill=cyan, color=cyan, opacity=0.5] (0,-1,2) -- (2,-1,2) -- (2,2,-1) -- (0,2,-1) -- cycle;    %piano // y
                                \draw (-0.5,0,1)--(2.5,0,1); 
                                    \draw[dashed] (-0.5,0,1)--(-1.1,0,1);
                                    \draw[dashed] (3,0,1)--(2.5,0,1);
                                \draw (-0.5,1,0)--(2.5,1,0); 
                                    \draw[dashed] (-0.5,1,0)--(-1.1,1,0);
                                    \draw[dashed] (3,1,0)--(2.5,1,0);
                            \end{tikzpicture}
                        \end{center}
                        \caption{$ax+cz+d=0$\\Piano parallelo all'asse $y$\\Interseca i piani $xy$ e $yz$ lungo\\ rette parallele all'asse $y$.}
                    \end{subfigure}
                    \begin{subfigure}{0.3\textwidth}
                        \begin{center}
                            \begin{tikzpicture}[scale=0.8]
                                \draw[help lines, color=white];
                                \draw[->,thin] (0,0,0)--(3,0,0) node[right]{$y$};
                                \draw[->,thin] (0,0,0)--(0,3,0) node[left]{$z$};
                                \draw[->,thin] (0,0,0)--(0,0,4) node[left]{$x$};
                            
                                \draw[fill=gray, color=gray, opacity=0.3] (0,0,0) -- (0,0,3.7) -- (2.7,0,3.7) -- (2.7,0,0) -- cycle;    %piano xy
                                \draw[fill=gray, color=gray, opacity=0.3] (0,0,0) -- (0,0,3.7) -- (0,2.7,3.7) -- (0,2.7,0) -- cycle;    %piano xz
                                \draw[fill=cyan, color=cyan, opacity=0.5] (2,-1,0) -- (2,-1,2) -- (-1,2,2) -- (-1,2,0) -- cycle;    %piano // x
                                \draw (1,0,-0.5)--(1,0,2.5); 
                                    \draw[dashed] (1,0,-0.5)--(1,0,-1.1);
                                    \draw[dashed] (1,0,3)--(1,0,2.5);
                                \draw (0,1,-0.5)--(0,1,2.5); 
                                    \draw[dashed] (0,1,-0.5)--(0,1,-1.1);
                                    \draw[dashed] (0,1,3)--(0,1,2.5);
                            
                            \end{tikzpicture}
                        \end{center}
                        \caption{$by+cz+d=0$\\Piano parallelo all'asse $x$\\Interseca i piani $xy$ e $xz$ lungo\\ rette parallele all'asse $x$.}
                    \end{subfigure}
                    \caption{Piani paralleli agli assi coordinati}
                \end{figure}
            \end{center}
            Se nell'equazione del piano due (soli) dei coefficienti $a$, $b$ o $c$ sono nulli, si ottiene un piano parallelo ad uno dei piani coordinati.
            \begin{center}
                \begin{figure}[h]
            
                    \begin{subfigure}{0.3\textwidth}
                        \begin{center}
                            \begin{tikzpicture}[scale=0.8]
                                \draw[help lines, color=white];
                                \draw[->,thin] (0,0,0)--(3,0,0) node[right]{$y$};
                                \draw[->,thin] (0,0,0)--(0,3,0) node[left]{$z$};
                                \draw[->,thin] (0,0,0)--(0,0,4) node[left]{$x$};
                            
                                \draw[fill=gray, color=gray, opacity=0.3] (0,0,0) -- (0,2.7,0) -- (2.7,2.7,0) -- (2.7,0,0) -- cycle;    %piano yz
                            
                                \draw[fill=cyan, color=cyan, opacity=0.5] (0,0,2) -- (0,2.7,2) -- (2.7,2.7,2) -- (2.7,0,2) -- cycle;    %piano yz
            
                                \node at (0,0,2){\textbullet};
                                    \node at (0.7,-0.5,2){$(k,0,0)$};
                                
                                \node at (0.7,1,0){$x=k$};
            
                            \end{tikzpicture}
                        \end{center}
                        \caption{Piano parallelo al piano $yz$}
                    \end{subfigure}
                    \begin{subfigure}{0.3\textwidth}
                        \begin{center}
                            \begin{tikzpicture}[scale=0.8]
                                \draw[help lines, color=white];
                                \draw[->,thin] (0,0,0)--(3,0,0) node[right]{$y$};
                                \draw[->,thin] (0,0,0)--(0,3,0) node[left]{$z$};
                                \draw[->,thin] (0,0,0)--(0,0,4) node[left]{$x$};
                            
                                \draw[fill=gray, color=gray, opacity=0.3] (0,0,0) -- (0,0,3.7) -- (0,2.7,3.7) -- (0,2.7,0) -- cycle;    %piano xz
                                \draw[fill=cyan, color=cyan, opacity=0.5] (2,0,0) -- (2,0,3.7) -- (2,2.7,3.7) -- (2,2.7,0) -- cycle;    %piano xz
                                \node at (2,0,0){\textbullet};
                                    \node at (2.7,-0.5,0){$(0,k,0)$};
                                \node at (1.3,1,0){$y=k$};
                            \end{tikzpicture}
                        \end{center}
                        \caption{Piano parallelo al piano $xz$}
                    \end{subfigure}
                    \begin{subfigure}{0.3\textwidth}
                        \begin{center}
                            \begin{tikzpicture}[scale=0.8]
                                \draw[help lines, color=white];
                                \draw[->,thin] (0,0,0)--(3,0,0) node[right]{$y$};
                                \draw[->,thin] (0,0,0)--(0,3,0) node[left]{$z$};
                                \draw[->,thin] (0,0,0)--(0,0,4) node[left]{$x$};
                            
                                \draw[fill=gray, color=gray, opacity=0.3] (0,0,0) -- (0,0,3.7) -- (2.7,0,3.7) -- (2.7,0,0) -- cycle;    %piano xy
                                \draw[fill=cyan, color=cyan, opacity=0.5] (0,2,0) -- (0,2,3.7) -- (2.7,2,3.7) -- (2.7,2,0) -- cycle;    %piano xy
                                \node at (0,2,0){\textbullet};
                                    \node at (-1,2,0){$(0,0,k)$};
                                \node at (0.7,1.2,0){$z=k$};
            
                            \end{tikzpicture}
                        \end{center}
                        \caption{Piano parallelo al piano $xy$}
                    \end{subfigure}
                    \caption{Piani paralleli ai piani coordinati}
                \end{figure}
            \end{center}
        \subsection{Equazione del piano passante per tre punti}
            Per individuare l'equazione del piano individuato da tre punti è sufficiente sostituire le coordinate dei punti nell'equazione del piano e risolvere il sistema. Siccome si tratta di un sistema di tre equazioni in 4 incongnite, alla fine si riuscirà a scrivere tre incognite in funzione dell'altra. Per quest'ultima andrà quindi scelto un valore arbitrario diverso da 0.
            \begin{ex}
                Scrivere l'equazione del piano passante per i punti $A(1,0,2)$, $B(0,1,3)$ e $C(0,0,3)$
            \end{ex}
                Consideriamo l'equazione del piano in forma cartesiana \[\pi:ax+by+cz+d=0\] e sostituiamo al suo interno le coordinate dei punti:
                \[\begin{array}{l}
                    A(1,0,2)\\
                    B(0,1,3)\\
                    C(0,0,3)
                \end{array} \left\{
                \begin{array}{l}
                    a(1)+b(0)+c(2)+d=0\\
                    a(0)+b(1)+c(3)+d=0\\
                    a(0)+b(0)+c(3)+d=0
                \end{array} \right.
                \]
                Ora risolviamo il sistema:
                \[\left\{\begin{array}{l}
                    a+2c+d=0\\
                    b+3c+d=0\\
                    3c+d=0
                \end{array} \right.\]
                \[\left\{\begin{array}{l}
                    a=-2c-d\\
                    b-d+d=0\\
                    3c=-d
                \end{array} \right.\]
                \[\left\{\begin{array}{l}
                    a=-2c-d\\
                    b=0\\
                    c=-\frac{1}{3}d
                \end{array} \right.\]
                \[\left\{\begin{array}{l}
                    a=\frac{2}{3}d-d=-\frac{1}{3}d\\
                    b=0\\
                    c=-\frac{1}{3}d
                \end{array} \right.\]
                Per comodità scelgo di porre $d=-3$, ma potrei scegliere un qualsiasi altro numero.
                \[\left\{\begin{array}{l}
                    a=-\frac{1}{3}(-3)=1\\
                    b=0\\
                    c=-\frac{1}{3}(-3)=1
                    d=-3
                \end{array} \right.\]
                \[\pi:x+z-3=0\]
        \subsection{Piani paralleli e perpendicolari}
            Siccome un piano è individuato dal suo vettore normale, esso può essere utilizzato anche per descriverne il parallelismo o la perpendicolarità con altri piani. In particolare:
            \begin{itemize}
                \item i vettori normali di due piani paralleli sono paralleli (CNS)
                \item i vettori normali di due piani perpendicolari sono perpendicolari (CNS)
            \end{itemize}
            \begin{figure}[h!]
                \centering
                \begin{subfigure}{0.49\textwidth}
                    \begin{center}
                        \begin{tikzpicture}[scale=0.7]
            
                            \draw[->,thin] (1.35,0,1.85)--(1.35,1,1.85) node[left]{$\overrightarrow{n}$};
                            \draw[->,thin] (1.85,1.5,1.85)--(1.85,2.5,1.85) node[left]{$\overrightarrow{n}'$};
                        
                            \draw[fill=gray, color=gray, opacity=0.3] (0,0,0) -- (0,0,3.7) node[left]{$\alpha$} -- (2.7,0,3.7) -- (2.7,0,0) -- cycle ; 
                            \draw[fill=cyan, color=cyan, opacity=0.5] (0.5,1.5,0) -- (0.5,1.5,3.7) node[left]{$\alpha'$} -- (3.2,1.5,3.7) -- (3.2,1.5,0) -- cycle;
                        
                        \end{tikzpicture}
                    \end{center}
                    \caption{Piani paralleli: $\overrightarrow{n}'=k\overrightarrow{n}$\\ $\frac{a}{a'}=\frac{b}{b'}=\frac{c}{c'}$}
                \end{subfigure}
                \begin{subfigure}{0.49\textwidth}
                    \begin{center}
                        \begin{tikzpicture}[scale=0.7]
            
                            \draw[->,thin] (0.85,1.5,1.85)--(0.85,2.5,1.85) node[left]{$\overrightarrow{n}'$};
                            \draw[->,thin] (2,2.5,1.85)--(3,2.5,1.85) node[above]{$\overrightarrow{n}$};
                        
                            
                            \draw[fill=gray, color=gray, opacity=0.5] (2,0,0) -- (2,0,3.7) node[left]{$\alpha$} -- (2,2.7,3.7) -- (2,2.7,0) -- cycle;
                            \draw[fill=cyan, color=cyan, opacity=0.5] (0.5,1.5,0) -- (0.5,1.5,3.7) node[left]{$\alpha'$} -- (3.2,1.5,3.7) -- (3.2,1.5,0) -- cycle;
                        
                        \end{tikzpicture}
                    \end{center}
                    \caption{Piani perpendicolari: $\overrightarrow{n}\cdot\overrightarrow{n}'$\\ $aa'+bb'+cc'=0$}
                \end{subfigure}
                \caption{Piani paralleli e perpendicolari}
            \end{figure}
        \subsection{Posizioni reciproche tra due piani}
            Dati due pani $\alpha$ e $\alpha'$ di equazioni $ax+by+cz+d=0$ e $a'x+b'y+c'z+d'=0$, essi sono
            \begin{itemize}
                \item Paralleli distinti se il sistema delle loro equazioni non ammette soluzioni, ovvero $\frac{a}{a'}=\frac{b}{b'}=\frac{c}{c'}\neq\frac{d}{d'}$
                \item Secanti se il sistema delle loro equazioni ammette infinite soluzioni, tutte appartenenti ad una retta
                \item Paralleli coincidenti se il sistea delle loro equazioni è verificato per ogni terna $(x,y,z)$ che soddisfa una delle due equazioni, ovvero $\frac{a}{a'}=\frac{b}{b'}=\frac{c}{c'}=\frac{d}{d'}$
            \end{itemize}
            \begin{center}
                \begin{figure}[h!]
                    \begin{subfigure}{0.3\textwidth}
                        \begin{center}
                            \begin{tikzpicture}[scale=0.8]
            
                                \draw[fill=gray, color=gray, opacity=0.3] (0,0,0) -- (0,0,3.7) node[left]{$\alpha$} -- (2.7,0,3.7) -- (2.7,0,0) -- cycle ; 
                                \draw[fill=cyan, color=cyan, opacity=0.5] (0.5,1.5,0) -- (0.5,1.5,3.7) node[left]{$\alpha'$} -- (3.2,1.5,3.7) -- (3.2,1.5,0) -- cycle;
                            
                            \end{tikzpicture}
                        \end{center}
                        \caption{Piani paralleli distinti}
                    \end{subfigure}
                    \begin{subfigure}{0.3\textwidth}
                        \begin{center}
                            \begin{tikzpicture}[scale=0.8]
                
                                \draw[thin] (2,1.56,0)--(2,1.56,3.7);
                                \draw[dashed,thin] (2,1.56,0)--(2,1.56,-1);
                                \draw[dashed,thin] (2,1.56,4.7)--(2,1.56,3.7);
                                
                                \draw[fill=gray, color=gray, opacity=0.5] (2,0,0) -- (2,0,3.7) node[left]{$\alpha$} -- (2,2.7,3.7) -- (2,2.7,0) -- cycle;
                                \draw[fill=cyan, color=cyan, opacity=0.5] (0.5,1,0) -- (0.5,1,3.7) node[left]{$\alpha'$} -- (3.2,2,3.7) -- (3.2,2,0) -- cycle;
                            
                            \end{tikzpicture}
                        \end{center}
                        \caption{Piani secanti}
                    \end{subfigure}
                    \begin{subfigure}{0.3\textwidth}
                        \begin{center}
                            \begin{tikzpicture}[scale=0.8]
            
                                \draw[color=gray, opacity=0.7] (0.5,1.5,0) -- (0.5,1.5,3.7) node[left]{$\alpha$} -- (3.2,1.5,3.7) -- (3.2,1.5,0) -- cycle;
                                \draw[fill=cyan, color=cyan, opacity=0.5] (0.5,1.5,0) -- (0.5,1.5,3.7) node[below]{$\alpha'$} -- (3.2,1.5,3.7) -- (3.2,1.5,0) -- cycle;
                            
                            \end{tikzpicture}
                        \end{center}
                        \caption{Piani paralleli coincidenti}
                    \end{subfigure}
                    \caption{Posizioni reciproche di due piani}
                \end{figure}
            \end{center}
        \subsection{Area di un triangolo nello spazio}
            Dato un triangolo di vertici $A(x_A,y_A,z_A)$, $B(x_B,y_B,z_B)$, $C(x_C,y_C,z_C)$, la sua area è data dalla formula 
            \[A=\frac{1}{2}\sqrt{\left|\begin{array}{ccc}
                x_A & y_A & 1\\
                x_B & y_B & 1\\
                x_C & y_C & 1
            \end{array}\right|^2
            \left|\begin{array}{ccc}
                y_A & z_A & 1\\
                y_B & z_B & 1\\
                y_C & z_C & 1
            \end{array}\right|^2
            \left|\begin{array}{ccc}
                x_A & z_A & 1\\
                x_B & z_B & 1\\
                x_C & z_C & 1
            \end{array}\right|^2
            }\]
    \section{Rette}
            \subsection{Equazione parametrica}
            \begin{wrapfigure}{r}{0.3\textwidth}
                \begin{center}
                    \begin{tikzpicture}[scale=0.8]
                        \draw[help lines, color=white];
                        \draw[->,thin] (0,0,0)--(3,0,0) node[right]{$y$};
                        \draw[->,thin] (0,0,0)--(0,3,0) node[left]{$z$};
                        \draw[->,thin] (0,0,0)--(0,0,4) node[left]{$x$};
                
                        \draw[dashed] (-0.6,0,0)--(0,0,0);
                        \draw[dashed] (0,-0.6,0)--(0,0,0);
                        \draw[dashed] (0,0,-1)--(0,0,0);
                
                        \coordinate (v) at (1,2,1);
                
                        \draw[->,thick, color=red] (0,0,0)--(v) node[right]{$\overrightarrow{v}$};
                
                        \coordinate (P0) at (2,0,1);
                        \node at ($(P0)+(0.5,0,0)$) {$P_0$};
                        \coordinate (P) at (3,2,2);
                        \node at ($(P)+(-0.5,0,0)$) {$P$};
                        
                        \fill[black] (P0) circle (1.5pt);
                        \fill[black] (P) circle (1.5pt);
                        
                        \draw[thin] ($(P0)-(.6,1.2,.6)$) -- ($(P)+(.7,1.4,.7)$) node[right]{$r$};
                        \draw[thin, dashed] ($(P0)-(.6,1.2,.6)$) -- ($(P0)-(v)$);
                        \draw[thin, dashed] ($(P)+(v)$) -- ($(P)+(.7,1.4,.7)$);
                
                        \draw[->, thick, color=red] (P0) -- (P) node[right]{$\overrightarrow{P_0P}$};
                        
                    \end{tikzpicture}
                \end{center}
                \caption{Definizione di una retta nello spazio}
            \end{wrapfigure}
            Una retta nello spazio è definita se si conoscono un suo punto $P_0(x_0,y_0,z_0)$ e un vettore $\overrightarrow{v}(l,m,n)$ che ne identifica la direzione. Allora un punto generico $P(x,y,z)$ appartiene alla retta se e solo se \[\overrightarrow{P_0P}(x-x_0,y-y_0,z-z_0)\parallelo\overrightarrow{v}(l,m,n)\]
            ovvero
            \[\overrightarrow{P_0P}=k\overrightarrow{v}~~~\text{con }k\in\mathbb{R}\]
            \[\left\{\begin{array}{l}
                x-x_0=kl\\
                y-y_0=km\\
                z-z_0=kn
            \end{array}\right.~~~\rightarrow~~~~
            \left\{\begin{array}{l}
                \mathbf{x=x_0+kl}\\
                \mathbf{y=y_0+km}\\
                \mathbf{z=z_0+kn}
            \end{array}\right.~~~\text{con }k\in\mathbb{R}\]
            \begin{definition}
                I coefficienti $l$, $m$, $n$ si dicono \textbf{coefficienti direttivi} perchè determinano la direzione della retta.
            \end{definition}
            \begin{definition}
                Il vettore $\overrightarrow{v}(l,m,n)$ si chiama \textbf{vettore direzione}.
            \end{definition}
            \subsubsection{Rette su GeoGebra}
                GeoGebra utilizza un formato particolare per visualizzare le equazioni delle rette:
                \[r: \,X =(x_0,y_0,z_0)+\lambda(l,m,n)\]
                in cui $(x_0,y_0,z_0)$ sono le coordinate di un punto appartenente alla retta e $l$, $m$ e $n$ sono i coefficienti direttivi.
        \subsection{Equazione cartesiana}
            Se tutti i coefficienti direttivi sono \textbf{non nulli}, si può scrivere l'equazione della retta in forma cartesiana:
            
            \[\left\{\begin{array}{l}
                x=x_0+kl\\
                y=y_0+km\\
                z=z_0+kn
            \end{array}\right.~~~
            \left\{\renewcommand{\arraystretch}{2}\begin{array}{ll}
                \dfrac{x-x_0}{l}=k&l\neq0\\
                \dfrac{y-y_0}{m}=k&m\neq0\\
                \dfrac{z-z_0}{n}=k&n\neq0
            \end{array}\right.\]

            \[\mathbf{\dfrac{x-x_0}{l}=\dfrac{y-y_0}{m}=\dfrac{z-z_0}{n}}~~~\text{equazione cartesiana della retta}\]
        \subsection{Equazione come intersezione di due piani}
            Una retta può essere determinata anche tramite l'intersezione di due piani distinti e non paralleli
            \[\left\{\renewcommand{\arraystretch}{1.5}\begin{array}{l}
                ax+by+cz+d=0\\
                a'x+b'y+c'z+d'=0
            \end{array}\right.\]
        \subsection{Equazione della retta passante per due punti}
            La retta passante per due punti qualsiasi $A(x_A,y_A,z_A)$ e $B(x_B,y_B,z_B)$ ha vettore direzione \[\overrightarrow{AB}(x_B-x_A, y_B-y_A, z_B-z_A)\]. La sua equazione può essere quindi individuata come visto ai punti precedenti:
            \[\frac{x-x_A}{x_B-x_A}=\frac{y-y_A}{y_B-y_A}=\frac{z-z_A}{z_B-z_A}\]
            \subsubsection{Condizioni di allineamento}
                Per verificare se tre punti sono allineati è sufficiente determinare la retta passante per due di loro e sostituire le coordinate del terzo all'interno dell'equazione. I tre punti sono allineati se e solo se tutte le uguaglianze sono verificate. In particolare i punti $A(x_A,y_A,z_A)$, $B(x_B,y_B,z_B)$ e $P(x_P,y_P,z_P)$ sono allineati se e solo se \[\frac{x_P-x_A}{x_B-x_A}=\frac{y_P-y_A}{y_B-y_A}=\frac{z_P-z_A}{z_B-z_A}\]
        \subsection{Fasci di piani aventi per asse una retta data}
        \begin{definition}
            Un fascio di piani è un insieme contenente tutti e soli i piani aventi per asse una retta data, detta asse del fascio. 
        \end{definition}
        \begin{wrapfigure}{r}{0.25\textwidth}
            \begin{center}
                \begin{tikzpicture}[scale=0.7]
        
                    \draw[thin] (2,1.56,0)--(2,1.56,3.7);
                    \draw[dashed,thin] (2,1.56,0)--(2,1.56,-1);
                    \draw[dashed,thin] (2,1.56,4.7) node[below]{$r$}--(2,1.56,3.7);
                    
                    \draw[fill=gray, color=gray, opacity=0.5] (2,0,0) -- (2,0,3.7) node[left]{$\alpha$} -- (2,2.7,3.7) -- (2,2.7,0) -- cycle;
                    \draw[fill=cyan, color=cyan, opacity=0.5] (0.5,1,0) -- (0.5,1,3.7) node[left]{$\alpha'$} -- (3.2,2,3.7) -- (3.2,2,0) -- cycle;
                    \draw[dashed] (0.5,2,0) -- (0.5,2,3.7) node[left]{$\alpha''$} -- (3.2,1,3.7) -- (3.2,1,0) -- cycle;
                
                \end{tikzpicture}
            \end{center}
            \caption{Fascio di piani di asse $r$}
        \end{wrapfigure}
        
            Se la retta è individuata come intersezione di due piani
            \[\left\{\renewcommand{\arraystretch}{1.5}\begin{array}{l}
                ax+by+cz+d=0\\
                a'x+b'y+c'z+d'=0
            \end{array}\right.\]
            Il fascio di piani può essere scritto per combinazione lineare come:
            \[ax+by+cz+d+k(a'x+b'y+c'z+d')=0\]
            Al variare del parametro $k$, l'equazione descrive tutti i piani appartenenti al fascio, fatta eccezione per il secondo piano generatore, ottenibile solo per $k=\pm\infty$, analogamente a quanto avviene per i fasci di rette
        \subsection{Posizioni reciproche di due rette}
            \subsubsection{Rette parallele}
                Date due rette $r$ e $s$ i cui vettori direzione sono rispettivamente $\overrightarrow{v}(l,m,n)$ e $\overrightarrow{w}(l',m',n')$, le due rette sono parallele se e solo se i loro vettori direzione sono paralleli:
                \[r\parallelo s ~~\Leftrightarrow~~ \overrightarrow{v}\parallelo\overrightarrow{w}~~\Leftrightarrow~~\overrightarrow{v}=k\overrightarrow{w}\text{, con }k\in\mathbb{R}~~\Leftrightarrow~~\frac{l}{l'}=\frac{m}{m'}=\frac{n}{n'}\]
                \subsubsection{Rette perpendicolari}
                \begin{wrapfigure}{r}{0.3\textwidth}
                    \begin{center}
                        \begin{tikzpicture}[scale=0.8]
    
                            \draw[fill=gray, color=gray, opacity=0.3] (0,0,0) -- (0,0,3.7) -- (2.7,0,3.7) -- (2.7,0,0) -- cycle;    %piano xy
    
                            \draw (0.8,0,1.2)--(0.8,1.5,1.2) node[left]{$r$};
                                \draw[dashed] (0.8,0,1.2)--(0.8,-1.1,1.2);
                                \draw[dashed] (0.8,2,1.2)--(0.8,1.5,1.2);
                                \draw[thick, ->, color=red] (0.8,0,1.2) -- (0.8,1,1.2)node[left]{$\overrightarrow{v}$};
                            \draw (2,0,1.5)--(2,1.5,1.5)node[left]{$s$};
                                \draw[dashed] (2,0,1.5)--(2,-1,1.5);
                                \draw[dashed] (2,2,1.5)--(2,1.5,1.5);
                                \draw[thick, ->, color=red](2,0,1.5)--(2,1,1.5)node[left]{$\overrightarrow{v}$};
                            \draw (1.05,0,0.7)--(0.3,0,2.2)node[below]{$t$}; 
                                \draw[dashed] (1.05,0,0.7)--(1.3,0,0.2);
                                \draw[dashed] (0.3,0,2.2)--(0.05,0,2.7);
                                \draw[thick, ->, color=blue] (0.8,0,1.2) -- (0.35,0,2.09)node[above]{$\overrightarrow{w}$};
                        \end{tikzpicture}
                    \end{center}
                    \caption{$r\parallelo s$ - $r\perp t$ - $s\perp t$\\$r$ e $t$ sono complanari\\$r$ e $s$ sono complanari\\$s$ e $t$ sono sghembe}
                \end{wrapfigure}
                Date due rette $r$ e $s$ i cui vettori direzione sono rispettivamente $\overrightarrow{v}(l,m,n)$ e $\overrightarrow{w}(l',m',n')$, le due rette sono perpendicolari se e solo se i loro vettori direzione sono perpendicolari:
                \[r\perp s ~~\Leftrightarrow~~ \overrightarrow{v}\perp\overrightarrow{w}~~\Leftrightarrow~~\overrightarrow{v}\cdot\overrightarrow{w}=0~~\Leftrightarrow~~ll'+mm'+nn'=0\]
            \subsubsection{Rette sghembe o incidenti}
                \begin{definition}
                    Due rette nello spazio si dicono \textbf{complanari} se e solo se appartengono allo stesso piano. In tal caso possono essere incidenti, parallele distinte o parallele coincidenti. 
                \end{definition}
                Se due rette hanno un punto in comune o sono parallele, sono complanari. 

                Se due rette sono complanari, allora sono incidenti o parallele.

                \begin{definition}
                    Due rette si dicono \textbf{sghembe} se non appartengono allo stesso piano. 
                \end{definition}
                Due rette sono sghembe se e solo se non hanno intersezioni e non sono parallele.
        \subsection{Rette e piani}
            Consideriamo un piano $\alpha$ con vettore normale $\overrightarrow{n}(a,b,c)$ non nullo e una retta $r$ con vettore direzione $\overrightarrow{v}(l,m,n)$ non nullo.
            \begin{figure}[h]
                \centering
                \begin{subfigure}{0.49\textwidth}
                    \begin{center}
                        \begin{tikzpicture}[scale=0.8]
                            \draw (-0.5,1,0)--(2.5,1,0); 
                                            \draw[dashed] (-0.5,1,0)--(-1.1,1,0);
                                            \draw[dashed] (3,1,0)--(2.5,1,0);
                
                            \draw[fill=gray, color=gray, opacity=0.3] (0,0,0) -- (0,0,3.7) node[left]{$\alpha$}-- (2.7,0,3.7) -- (2.7,0,0) -- cycle;    %piano xy
                
                                \draw[thick, ->, color=red](1.35,0,1.85)--(1.35,1,1.85)node[left]{$\overrightarrow{n}$};
                            
                            \draw (-0.5,1,0)--(2.5,1,0) node[above]{$r$}; 
                                \draw[dashed] (-0.5,1,0)--(-1.1,1,0);
                                \draw[dashed] (3,1,0)--(2.5,1,0);
                                \draw[thick, ->, color=red] (1,1,0) -- (2,1,0)node[above]{$\overrightarrow{v}$};
                        \end{tikzpicture}
                    \end{center}
                    \caption{$r\parallelo \alpha$}
                \end{subfigure}
                \begin{subfigure}{0.49\textwidth}
                    \begin{center}
                        \begin{tikzpicture}[scale=0.8]
                            \draw[fill=gray, color=gray, opacity=0.3] (0,0,0) -- (0,0,3.7) node[left]{$\alpha$}-- (2.7,0,3.7) -- (2.7,0,0) -- cycle;    %piano xy
                            %(y,z,x)
                            \draw (0.8,0,1.2)--(0.8,1.5,1.2) node[left]{$r$};
                                \draw[dashed] (0.8,0,1.2)--(0.8,-1.1,1.2);
                                \draw[dashed] (0.8,2,1.2)--(0.8,1.5,1.2);
                                \draw[thick, ->, color=red] (0.8,0,1.2) -- (0.8,1,1.2)node[left]{$\overrightarrow{v}$};
                
                                \draw[thick, ->, color=red](2,0,1.85)--(2,1,1.85)node[left]{$\overrightarrow{n}$};
                            
                        \end{tikzpicture}
                    \end{center}
                    \caption{$r\perp \alpha$}
                \end{subfigure}
                \caption{Posizioni reciproche tra retta e piano}
            \end{figure}
            \subsubsection{Retta parallela al piano}
                La retta e il piano sono paralleli se il vettore direzione della retta e il vettore normale del piano sono perpendicolari.
                \[r\parallelo\alpha ~~\Leftrightarrow~~\overrightarrow{n}\perp\overrightarrow{v}~~\Leftrightarrow~~ al+bm+cn=0\]
                In particolare se la retta e il piano hanno almeno un punto in comune, allora la retta giace sul piano. 
                \subsubsection{Retta perpendicolare al piano}
                La retta e il piano sono perpendicolari se il vettore direzione della retta e il vettore normale del piano sono paralleli.
                \[r\perp\alpha ~~\Leftrightarrow~~\overrightarrow{n}\parallelo\overrightarrow{v}~~\Leftrightarrow~~ \frac{a}{l}=\frac{b}{m}=\frac{c}{n}~~\left(\Leftrightarrow~~ \frac{l}{a}=\frac{m}{b}=\frac{n}{c}\right)\]
    \section{Distanze}
        \subsection{Distanza di un punto da un piano}
            Dato il piano \[\alpha:ax+by+cz+d=0\] e il punto $A(x_A,y_A,z_A)$, la misura della distanza tra il punto e il piano è data da
            \[d(A,\alpha)=\frac{|ax_A+by_A+cz_A+d|}{\sqrt{a^2+b^2+c^2}}\]
        \subsection{Distanza fra due piani paralleli}
            Si considerino due piani paralleli $\alpha$ e $\beta$ , la distanza tra esi è congruente alla distanza tra il piano $\beta$     e un punto $P\in\alpha$ qualsiasi. 
        \subsection{Distanza di un punto da una retta}
            \textbf{NON ESISTE UNA FORMULA SPECIFICA}

            Si considerino una retta r $r$ e un punto $P$ non appartenente alla retta. Un metoto può essere quello di determinare il punto $H$ tale che $PH\perp r$ e calcolare a questo punto la distanza $PH$. Per fare questo è necassario determinare l'equazione del piano perpendicolare a $r$ e passante per $P$ e poi trovare la sua intersezione $H$ con la retta:

            \begin{ex}
                Calcolare la distanza tra il punto $P(2,1,5)$ e la retta $r: \left\{\begin{array}{l}
                    x=-3+3t\\
                    y=2t\\
                    z=2+4t
                \end{array}\right.$
            \end{ex}
            \begin{enumerate}
                \item Determinare l'equazione del piano $\pi\perp r$ passante per $P$. Siccome la retta è perpendicolare al piano, il suo vettore direzione è parallelo al vettore direzione del piano.
                    \[\overrightarrow{v}(3,2,4)~~~~~~P(2,1,5)\]
                    \[3(x-2)+2(y-1)+4(z-5)=0\]
                    % \[3x-6+2y-2+4z-20=0\]
                    \[\pi: 3x-2y+4z-28=0\]
                \item Determinare il punto di intersezione $H$ tra la retta $r$ e il piano $\pi$
                    \[r\cap\pi\left\{\begin{array}{l}
                        x=-3+3t\\
                        y=2t\\
                        z=2+4t\\
                        3x-2y+4z-28=0
                    \end{array}\right.~~~~~~~~~~~~\renewcommand{\arraystretch}{2}\begin{array}{c}
                        3(-3+3t)-2(2t)+4(2+4t)-28=0\\
                        t=1
                    \end{array}\]
                    \[\left\{\begin{array}{l}
                        x=-3+3(1)=0\\
                        y=2(1)=2\\
                        z=2+4(1)=6
                    \end{array}\right.~~~~~~~~~~~~H(0,2,6)\]
                \item Determinare la lunghezza del segmento $\overline{PH}$, che ciuncide con la distanza del punto $P$ dalla retta $r$;
                    \[d(P,r)=\overline{PH}=\sqrt{(0-2)^2+(-2-1)^2+(6-5)^2}=\sqrt{6}\]
            \end{enumerate}
        \subsection{Distanza tra due rette parallele}
            \textbf{NON ESISTE UNA FORMULA SPECIFICA.}
            Per calcoloare la distanza tra due rette parallele $r$ e $s$ è sufficiente calcolare la distanza tra un punto $P\in r$ qualsiasi e la retta $s$.
        \subsection{Distanza tra due rette sghembe}
            \textbf{NON ESISTE UNA FORMULA SPECIFICA.}
            Per determinare la distanza tra due rette sghembe $r$ e $s$ è necessario calcolare la misura del segmento $\overline{RS}$, con $R$ e $S$ due punti appartenenti rispettivamente a $r$ e $s$ tali che $\overrightarrow{RS}\perp r$ e $\overrightarrow{RS}\perp s$.
            \begin{theorem}
                Prese due rette sghembe $r$ e $s$ nello spazio, esiste ed è unica la retta $t$ tale che $t\perp r$ e $t\perp s$
            \end{theorem}
    \section{Superfici notevoli}
        \subsection{Superficie sferica}
            \begin{definition}
                Si dice \textbf{superficie sferica} il luogo geometrico di tutti e soli i punti del piano che hanno la stessa distanza $r$ da un punto fisso detto centro $C(x_0,y_0,z_0)$.
            \end{definition}
            La sfera, come la circonferenza, è identificata in due modi diversi:
            \begin{enumerate}
                \item come luogo geometrico:
                    \[(x-x_0)^2+(y-y_0)^2+(z-z_0)^2=r^2\]
                \item forma canonica, che rappresenta una superficie sferica se e solo se $a^2+b^2+c^2-4d\geq 0$ (condizione di realtà):
                    \[x^2+y^2+z^2+ax+by+cz+d=0\] \[\text{con } C(-\frac{a}{2},-\frac{b}{2},-\frac{c}{2}) \text{ e } r=\frac{1}{2}\sqrt{a^2+b^2+c^2-4d}\]
            \end{enumerate}
            \subsubsection{posizioni reciproche tra piano e sfera}
            \begin{wrapfigure}{r}{0.3\textwidth}
                \begin{center}
                    \begin{tikzpicture}[scale=0.8]
                        \shade[ball color = gray!40, opacity = 0.4] (0,0) circle (2cm);
                        \draw (0,0) circle (2cm);
                        \draw (-2,0) arc (180:360:2 and 0.6);
                        \draw[dashed] (2,0) arc (0:180:2 and 0.6);
                        \fill[fill=black] (0,0) circle (1pt);
                        \draw[dashed] (0,0 ) node[above]{$C$} -- node[above]{$r$} (2,0);
                    \end{tikzpicture}
                \end{center}
                \caption{Superficie sferica di centro $C$ e raggio $r$}
            \end{wrapfigure}
            Dato un piano $\alpha$ e una superficie sferica di cerntro $C$ e raggio $r$, si ha che:
            \begin{itemize}
                \item Se $d(C,\alpha)<r$, il piano interseca la sfera lungo una circonferenza;
                \item Se $d(C,\alpha)=r$, il piano è tangente alla sfera in un punto $P$ e perpendicolare al raggio $CP$.
                \item Se $d(C,\alpha)>r$, il piano è esterno alla sfera.
            \end{itemize}
\end{document}